\documentclass{article}
\usepackage{listings}
\usepackage{color}

\definecolor{codegreen}{rgb}{0,0.6,0}
\definecolor{codegray}{rgb}{0.5,0.5,0.5}
\definecolor{codepurple}{rgb}{0.58,0,0.82}
\definecolor{backcolour}{rgb}{0.95,0.95,0.92}

\lstdefinestyle{mystyle}{
	backgroundcolor=\color{backcolour},   
	commentstyle=\color{codegreen},
	keywordstyle=\color{magenta},
	numberstyle=\tiny\color{codegray},
	stringstyle=\color{codepurple},
	basicstyle=\ttfamily\footnotesize,
	breakatwhitespace=false,         
	breaklines=true,                 
	captionpos=b,                    
	keepspaces=true,                 
	numbers=left,                    
	numbersep=5pt,                  
	showspaces=false,                
	showstringspaces=false,
	showtabs=false,                  
	tabsize=2
}

\lstset{style=mystyle}

\begin{document}
	
	\begin{titlepage}
		\centering
		\vspace*{1cm}
		\Large Universidad Nacional Autónoma de México\\
		\Large Facultad de Ingeniería\\
		\vspace*{0.5cm}
		\large Nombre del Alumno:\\
		\large Silverio Martínez Andrés \\
		\vspace*{0.5cm}
		\large Nombre de la Materia:\\
		\large Bases de Datos\\
		\vspace*{0.5cm}
		\large Nombre del Profesor:\\
		\large Fernando Arreola Franco \\
		\vspace*{0.5cm}
		\large Tarea 3:\\
		\large Creación de usuario en Postgres \\
		\vspace*{0.5cm}
		\large Grupo:\\
		\large 1 \\
		\vspace*{0.5cm}
		\large Semestre:\\
		\large 2024 - 2:\\
		\vspace*{0.5cm}

	\end{titlepage}
	
	\title{Tarea 3}
	\author{Silverio Martínez Andrés}
	\date{\today}
	
	\maketitle
	
	\section{Introducción}
Se quiere crear un usuario nuevo en postgres, donde este tenga un limite de conexiones, una contraseña unica y un mes de vigencia. Seguido a esto, se le busacará asignar un rol al usuario que se acaba de crear

\section{Código para Postgres}
De acuerdo al enunciado anterior, se hizo un pedazo de código para Postgres, el cual quedo de la siguiente manera:

\begin{lstlisting}[language=SQL]
	-- Crear usuario
	CREATE USER nuevo_usuario WITH 
	PASSWORD 'contrasena_segura' 
	VALID UNTIL '2024-03-14'
	CONNECTION LIMIT 10;
	
	-- Crear rol
	CREATE ROLE rol_estudiante;
	
	-- Asignar permisos al rol
	GRANT SELECT, UPDATE, DELETE ON estudiante TO rol_estudiante;
	
	-- Asignar rol al usuario
	GRANT rol_estudiante TO nuevo_usuario;
\end{lstlisting}

\section{Conclusión}
De acuerdo a lo poco que se abarcó acerca de la creacion de usuarios y la creación de roles en postgres, es probable que presente errores, pero para corroborar que si funciona, sería necesario correrlo en Postgres, cosa que, por el momento, desconozco como hacer.

\end{document}